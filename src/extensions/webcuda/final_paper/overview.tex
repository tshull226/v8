

talk about main features from a JavaScript perspective
\begin{itemize}
\item talk about different classes (Modules, Function, etc \ldots\\
maybe each of these should be subparagraphs
\item ability to use strings as CUDA code
\item ability to use shared memory
\item support many data types
\end{itemize}

\begin{table}
\begin{center}
\begin{tabular}{| l | l |}
\hline
Benchname & Input Size \\
\hline
random-pixel & N/A \\
\hline
Nbody &  1024 bodies \\
\hline
\end{tabular}
\end{center}
\caption{Main features of \name}
\label{webcudaSpec}
\end{table}

This section describes the \name JavaScript extension. We have modeled our
specification off similar projects \cite{webCL, nokiaCL, amdCL} in the OpenCL
domain.  In addition, the protocols provided by the CUDA Driver API
\cite{cudaAPI} has heavily influenced the structure and layout of our extension.
Below we highlight the salient features of the extension. Please refer to
Table~\ref{webcudaSpec} throughout the discussion as a brief reference of
various \name functionality. The full \name specification, generated by JSDOC
\cite{JSDOC}, can be found at
\url{http://tshull226.synology.me/CS598SVA/JSDoc/index.html}


%should only have 1, inclusive code example
\begin{figure*}
	\begin{center}
\begin{lstlisting}[frame=single]
var device = webcuda.Device(0);
var context = webcuda.Context(0, device);
var hostMem = new Int32Array(size);
var deviceMem = webcuda.memAlloc(hostMem.buffer.byteLength);
\end{lstlisting}
\end{center}
\caption{Simple \name Example}
\label{codeExample}
\end{figure*}

\paragraph{Context Creation} The CUDA Driver API dictates the steps necessary to connect to a CUDA-enabled
device and create an environment for kernel execution. While the process is
similar to native CUDA programming, many items, namely, context creation must be
done explicitly. {\bf why is context creation necessary?}
Figure \ref{codeExample} shows the protocol for retrieving a handle to a
CUDA-enabled device and creating a CUDA context. A simpler, 1 or 0 step context
creation is left as future work.
%A simpler, implicit connection is left as future work.

\paragraph{Memory Allocation} Memory within the current CUDA context can be
created through the \textit{webcuda.memAlloc()}. Similar to CUMALLOC() this
allow for creation of a memory region of an arbitrary byte size. {\bf what about %
alignment issues?}

\paragraph{Data Communication} \name supports two functions,
\textit{webcuda.CopyDtoH} and \textit{webcuda.CopyHtoD} for the transferring of data
between the host and device. Our specification requires the JavaScript Object
to be a Typed Array \cite{typearray}. Typed Arrays are a feature of the new "harmony"
specification \cite{harmony}. Possible Types include UInt32, Float32, MORE.

\paragraph{Kernel Execution}
%this has to include function loading, compiling, launching, etc...

\paragraph{Freeing Resources}

