

\begin{table}
\begin{center}
\begin{tabular}{| l | p{5.5cm} | }
\hline
Function Name & Brief Description \\
\hline
Device & Retrieve Handle to CUDA-enabled device \\
\hline
Context & Setup CUDA context on specific device \\
\hline
ctxFree & Free resources consumed by specified CUDA Context\\
\hline
memAlloc & Allocate CUDA device memory \\
\hline
copyHtoD  & Copy Host Memory to CUDA Device \\
\hline
copyDtoH & Copy CUDA Device Memory to Host \\
\hline
free & Free CUDA Device Memory \\
\hline
compileFile & Compile and Load specified .cu file \\
\hline
moduleLoad & Load specified .cubin or .ptx file \\
\hline
moduleUnload & Free resources consumed by specified CUDA module\\
\hline
launchKernel & Launch CUDA kernel\\
\hline
synchronizeCtx & Block until all CUDA device operations in current context are complete \\
\hline
\end{tabular}
\end{center}
\caption{Main features of \name}
\label{webcudaSpec}
\end{table}

This section describes the \name JavaScript extension. We have modeled our
specification off similar projects \cite{webCL, safariCL, nokiaCL, chromeCL} in the OpenCL
domain.  In addition, the protocols provided by the CUDA Driver API
\cite{cudaAPI} has heavily influenced the structure and layout of our extension.
Below we highlight the salient features of the extension. Please refer to
Table~\ref{webcudaSpec} throughout the discussion as a brief reference of
various \name functionality. The full \name specification, generated by JSDOC
\cite{JSDOC}, can be found at
\url{http://tshull226.synology.me/CS598SVA/JSDoc/index.html}
Below we walk through the code example shown in Figure~\ref{codeExample}, highlighting
the salient features of \namens.


%should only have 1, inclusive code example
%copied off stack exchange as a language format for JavaScript
\lstdefinelanguage{HTML5}{
	sensitive=true,
	keywords={%
		% JavaScript
		typeof, new, true, false, catch, function, return, null, catch,
		switch, var, if, in, while, do, else, case, break,
		% HTML
		html, title, meta, style, head, body, script, canvas,
		% CSS
		border:, transform:, -moz-transform:,
		transition-duration:, transition-property:,
		transition-timing-function:
	},
	% http://texblog.org/tag/otherkeywords/
	otherkeywords={<, >, \/},   
	ndkeywords={class, export, boolean,
	throw, implements, import, this},   
	comment=[l]{//},
	% morecomment=[s][keywordstyle]{<}{>},  
	morecomment=[s]{/*}{*/},
	morecomment=[s]{<!}{>},
	morestring=[b]',
	morestring=[b]",    
	alsoletter={-},
	alsodigit={:}
}
\lstset{ language=HTML5, numbers=left, stepnumber=1 }
\begin{figure*}
	\begin{center}
		\small
		\lstinputlisting{example.js}
	\end{center}
	\caption{Simple \name Example}
	\label{codeExample}
\end{figure*}

\paragraph{Context Creation} The CUDA Driver API dictates the steps necessary to
connect to a CUDA-enabled device and create an environment for kernel execution.
While the process is similar to native CUDA programming, many items, namely,
context creation must be done explicitly. A CUDA context is associate with a
specific process thread and is used to differentiate between possibly multiple
threads interacting with a CUDA-enabled device simultaneously. After a context
is created for a thread, the CUDA drivers are able to associate all memory
allocations to a specific device, thus allow different processes to have
different unshared address spaces.  A simpler, implicit context creation, as is
done in the CUDA runtime, is left as future work.

 Figure~\ref{codeExample}, lines 2-7 show the protocol for
retrieving a handle to a CUDA-enabled device and creating a CUDA context. Since
a context is associated with a specific CUDA device, a handle to a CUDA-enabled
device must first be attained. Passed along with the handle to the device is a
flag indicating any non-default features this CUDA context is to have.

\paragraph{Memory Allocation} Memory within the current CUDA context can be
created through the \textit{webcuda.memAlloc()}. Similar to cuMemAlloc() this
function allow for creation of a memory region of an arbitrary byte size. Lines
8 and 9 in Figure~\ref{codeExample} show the creation of device and host memory.
Our specification requires host memory JavaScript Object to be Typed Arrays
\cite{typedarray} to communicate with device memory. Typed Arrays are a feature
of the new "harmony" specification \cite{harmony}. Possible Types include
UInt32, Float32, and many more. While typical JavaScript can hold arbitrary type
of Objects and must be allocated accordingly, Type Arrays enforce data sizes and
therefore can be allocated the same way as C arrays. This allows for efficient
communication between the host and device when copying memory.


\paragraph{Data Communication} \name supports two functions,
\textit{webcuda.CopyDtoH} and \textit{webcuda.CopyHtoD}, for transferring 
data between the host and CUDA-enabled device. Lines 12 and 33 show examples of copying
memory from the host to device and vice versa. As described in the previous
paragraph, Type Arrays are used to maximize communication efficiency. Currently,
only synchronous memory transfers are allow, but subsequent work can implement
support for asynchronous memory accesses.

\paragraph{Kernel Setup}
Lines 14-16 show the compilation of a CUDA file and the extraction of a kernel
from the module. The CUDA Driver API specifies that a CUDA module must first be
loaded and the kernel function retrieved from the module before a kernel can be
launched. \name supports three ways of loading a CUDA module: loading a
pre-compiled .cubin or .ptx file, performing on the fly compilation and loading
of a .cu file, or performing on the fly compilation of a JavaScript string. In
this example on line 14 we demonstrate the on the fly compilation of a .cu
file. This loaded module is regular CUDA code and can contain many CUDA kernels.
Therefore, the specific kernel one wants to launch must also be queried, as show
on line 16. \name supports the loading and execution of multiple CUDA kernels
from the same or different files.

\paragraph{Kernel Execution}
Lines 18-28 show the execution of the kernel. The grid and block dimensions must
be passed to the kernel as arrays of Integers. Note that in the current
implementation of \namens, all dimensions must explicitly be given a value,
unlike the CUDA runtime where unfilled dimensions are implicitly assigned the
value 1. \name has support for shared memory and the amount required must be
given in number of bytes. Finally, parameters are passed to the kernel in an
array of object. Each element in the array must be an object with a property
name describing the type of element to be passed. Allowable property names are
\textit{memParam}, \textit{intParam}, \textit{floatParam}, and
\textit{doubleParam} along with the appropriate value type. The elements in the
array must be passed in the same order as expected by the kernel function.

The launching of the CUDA array, performed by \textit{webcuda.launchKernel} is
done asynchronously. Therefore, it is recommended to call
\textit{webcuda.synchronizeCtx} to block the host until all device operations
have completed.

\paragraph{Freeing Resources}
After the CUDA kernel has completed and the results have been transferred to the
host (Line 32), the final step is to free the resources acquired during the
kernel launch. Lines 34-36 show the \name calls to release the memory, module,
and context resources allocated.

The process shown above should be very similar regardless of the kernels being
run. The main differences should be the type of memory used, the amount of
memory used, and the number of kernels executed. We hope by walking through a
simple program and explaining key \name features, one has enough of an
understanding to begin programming in \namens.

