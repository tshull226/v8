
The idea of enabling GPGPU programming within more domains has attracted much
interest from the web community. Below we mention the projects we feel are most relevant
to \namens.

%\paragraph{node-cuda}
The project most similar to \name is node-cuda~\cite{nodeCuda}. The goal of
this project is to bring CUDA bindings to JavaScript execution within
Node.js~\cite{nodeJS}. Like Chrome, Node.js uses the V8 compiler. However,
their project, and the current V8 version used in Node.js, is a version of V8
released the major overhaul of the external API discussed in~\ref{challenges}.
Therefore, while this project was extremely helpful in identifying core CUDA
Driver API features the \name spec needed to cover, the code itself was not
useful, as we use the newest version of V8.

%\paragraph{WebCL}
Several previous projects have been started to implement WebCL \cite{webCL} in
various browsers, including Safari \cite{safariCL} and Firefox \cite{nokiaCL},
and Chromium \cite{chromeCL}.  WebCL, discussed in Section~\ref{webCLDisc}, is
project started by the Khronos Group, to add OpenCL-like bindings to browsers.
Since the start of our CS598 project, AMD released a WebCL implementation on
the Chromium browser \cite{chromeCL}. We hope to eventually compare \name to
the Chrome-based WebCL over a variety of benchmarks to determine the
effectiveness of each.

%\paragraph{ParallelJS}
Another project that has been released since the start of our CS598 project is
\textit{ParallelJS}~\cite{parallelJS}. The aim of this project closely mirrors
what we in this paper (as well as in previous progress reports written before we
learned of this publication) suggest as future work in Section~\ref{future}, namely, to create a
runtime environment for JavaScript that can recognize native JavaScript code amenable to
execution on a GPU and perform the necessary source-to-source translation to run
the code on the GPU. This paper is a good first step in this direction, but
they only support very specific JavaScript code patterns so there is still much
work to be done in this area.

%\paragraph{JCuda, Python Cuda}
Many other projects are in development to bring CUDA extensions to languages
beyond C/C++ and Fortran, including JCUDA~\cite{jcuda} for the Java domain and
PyCUDA~\cite{pycuda} for the Python domain.  Similar to \namens, these projects
wrap CUDA functions within methods within their respective domains. PyCUDA
focuses on exposing the CUDA Drive API, like \namens, while JCUDA exposes both
the CUDA Runtime and Driver APIs.


