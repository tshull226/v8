
%( used for spell checking
%(
We have implemented our \name standard within Google's V8 \cite{V8website}
compiler.   We chose to use the Google V8  compiler for our implementation due
to prior experience with the compiler that provided us with a solid
understanding on the compiler's internals. In addition, V8 is considered a state
of the art JavaScript JIT compiler and is extremely prevalent due to its
inclusion in the Chromium browser \cite{chromium} and other projects such as
NodeJS \cite{nodeJS} and V8.NET \cite{V8.NET}. We have extended V8 through its
external API and integrated the extension in d8, V8's standalone JavaScript
execution engine. 

In total, our implementation is slightly more than 800 LOC, and is very
self-contained. By using the external V8 API, no modifications (except for
profiling purposes) had to be made to V8's code base. By not modifying internal
structures, there is no effect on the performance of JavaScript and native
JavaScript runs in our implementation without any additional overheads.

Below, we provide a high-level overview and explanation of main features of
\namens's implementation in V8.  Complete documentation for \namens's implementation,
generated with Doxygen \cite{doxygen}, can be found at
\url{http://tshull226.synology.me/CS598SVA/doxygen/index.html}. We
have also released our code to the project domain through GitHub \cite{github}.
Our repository can be found at
\url{https://github.com/tshull226/v8/tree/master/src/extensions/webcuda}.

At a high level, our implementation wraps main CUDA Driver API \cite{cudaAPI}
calls inside JavaScript wrappers that adhere to the \name specification. We
chose to use the CUDA Driver API, as opposed to the CUDA Runtime API
\cite{cudaRuntimeAPI}, so our extension can easily be compiled alongside d8
using the gyp \cite{gyp} configuration manager and gcc \cite{gcc}, instead of
needing to use nvcc \cite{nvcc}.

Our implementation is able to store handles to CUDA structs by wrapping them
within JavaScript objects. JavaScript objects are allowed to have internal
fields, inaccessible to programmers, that the compiler can use to store various
information about the object. Therefore, we wrap CUDA structs in a way such that
basic CUDA information is available to the user (such as whether the object was
successfully created), and internal fields contain the information
necessary for performing CUDA Driver API calls.

We tried to follow documented programming patterns for V8 embedded application
developers provided by various resources online \cite{embeddersGuide,
nodeJSDocumentation}. However, due to the rapidly evolving nature of the V8
source code, the external API has drastically changed and even Google's
documentation \cite{embeddersGuide} is deprecated. Therefore, we found the best
documentation for using the API to be Google code samples provided along with
V8.

\lstset{ language=C++, numbers=left, stepnumber=1, tabsize=1, keepspaces=false,
	breaklines=true, escapeinside={\%}{*)}}
\begin{figure*}
	\begin{center}
		\small
		\lstinputlisting{exampleC1.cc}
\end{center}
\caption{Creating new \name Device Object}
\label{v8codea}
	\begin{center}
		\small
		\lstinputlisting{exampleC2.cc}
\end{center}
\caption{C++ Implementation of \namens's \textit{webcuda.free()} method}
\label{v8codeb}

\end{figure*}

Our implementation class structure closely mirrors the specification, with
function wrappers broken into files and classes based on their functionality.
Figures~\ref{v8codea} and \ref{v8codeb} show code examples of common routines
executed extensively throughout our code both to wrap CUDA structs in a
JavaScript Object and unwrap CUDA structs from a JavaScript Object for CUDA
Driver API calls.  

Figure~\ref{v8codea} shows the code for creating a new connection to a
CUDA-enabled device. This is a function wrapper for the \textit{webcuda.Device()}
function in the \name specification. The user provides as input the device
number and expects the function to return a cuDevice struct wrapped in a
JavaScript Object.  The wrapper method receives only 1 parameter of type
\textit{FunctionCallBackInfo}. This parameter contains all essential information
about the function's caller, various arguments passed to the function with
JavaScript, as well as a handle to the object which will be returned by this
function, which by default is of type undefined. 

Line 2 creates the HandleScope manager for this function.  HandleScopes are V8's
approach to managing garbage collection of JavaScript objects within a function. It
deletes all stack allocated objects upon exiting the function.  A more
thorough explanation of HandleScopes can be found in the V8 embedder's
guide~\cite{embeddersGuide}.  

Lines 4 and 5 retrieve the integer value given as
input to the function. Arguments passed to the JavaScript function are stored
as indexes in the argument received in the C++ domain. Since in JavaScript
Integer primitives must be wrapped for function calls, our code must extract the
integer value from the Object.

Lines 7 and 8 show the creation of the C++ Device object and the calling of the
CUDA Driver API method, \textit{cuDeviceGet}, that returns a pointer to a
cuDevice handle. We created the Device Class to store this value.

Lines 10 and 11 create the JavaScript object to store the C++ Device Object. A
global object, constructor\_template, is used as a basis for all JavaScript Device
Objects created. The constructor\_template object was created during
initialization. This object contains mappings to the various properties of the
JavaScript Device Object visible to the user as specified by \namens, including the
CUDA device's name, compute capacity, and amount of memory.

Line 13 creates a JavaScript wrapper for the C++ Device Object. This is
necessary for the Device Object to be stored as a internal field within the
returned JavaScript Object.

Line 15 stores the Device Object wrapper in an internal field of the JavaScript
Device Object and finally line 17 sets the return value to the JavaScript Device
Object.


Figure~\ref{v8codeb} shows the code implementing the \name feature
\textit{webcuda.free()}. This function frees the CUDA device memory pointed to by
the parameter passed to the function. 

As in the previous example, Line 2 creates a HandleScope manager for this
function.  Line 4 performs multiple functions. First, it retrieves the parameter
passed from the JavaScript function. Next, it casts the value of the pointer to
a JavaScript object. Finally, it calls another member function, UnwrapDevicePtr, a
function used to extract the C++ Mem Object from the JavaScript Object. This
process is repeated whenever a C++ class needs to be extracted from a JavaScript
Object, leading to much code reuse.

Line 6 calls the CUDA Driver API method, \textit{cuMemFree}, used to free the
device memory. Finally, Line 8 sets the function's return value to the result of
the CUDA Driver Call so a programmer knows whether the call was able to
successfully free the CUDA-device memory or not.


As can be seen through the code examples above, our implementation of \name is intuitive,
straightforward, and can be easily extended in the future as more features are
included within the \name specification.


