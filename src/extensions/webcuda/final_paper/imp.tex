

(
We have implemented our \name standard within google's v8 \cite{v8} compiler. We
have also released our code to the project domain through Github \cite{github}.
Our repository can be found at
\url{https://github.com/tshull226/v8/tree/master/src/extensions/webcuda}.  We chose to use
the google v8  compiler for our implementation due to prior experience with the
compiler that provided us with a solid understanding on the compiler's
internals. In addition, v8 is considered a state of the art JavaScript JIT
compiler and is extremely prevalent due to its inclusion in the Chrome browser
\cite{chrome} and other projects such as NodeJS \cite{nodeJS} and V8.NET
\cite{V8.NET}. We have extended V8 through its external API and integrated the
extension in d8, v8's standalone JavaScript execution engine. 

In total, our implementation is slightly more than 800 LOC, but is very
self-contained. By using the external V8 API, no modifications (except for
profiling purposes) had to be made to v8's code base. By not modifying internal
structures, there is no effect on the performance of JavaScript and native
JavaScript runs in our implementation without any additional overhead.

Documentation for the full \name implementation within v8,
generated with Doxygen \cite{doxygen}, can be found at
\url{http://tshull226.synology.me/CS598SVA/doxygen/index.html}

At a high level, our implementation wraps main CUDA Driver API
\cite{cudaDriverAPI} calls inside JavaScript wrappers that adhere to the \name
specification. We had to use the CUDA Driver API, as opposed to the CUDA Runtime
API \cite{cudaRuntimeAPI}, due to the necessity of compiling our extension along
with d8 using the gyp \cite{gyp} configuration manager and gcc
\cite{gcc} instead of using nvcc \cite{nvcc}.

Our implementation is able to store handles to CUDA structs through wrapping them
within JavaScript objects. JavaScript objects are allowed to have internal
fields, inaccessible to programmers, that the compiler can use to store various
information about the object. Therefore, we wrap CUDA structs in a way such that
based information is available to the user (such as whether the object was
successfully created) and our extension is able to unwrap the information
necessary for performing CUDA Driver API calls.

We tried to follow documented programming patterns for embedded application
developers according to various resources online
\cite{embeddersGuide, nodeJSDocumentation}. However, due to the rapidly evolving
nature of the v8 source code, the external API has drastically changed and even
google's documentation \cite{embeddersGuide} is deprecated. Therefore we found
the best documentation for using the API to be google code samples provided
along with v8.

Our implementation class structure closely mirrors the specification, with
functions broken into respective groups based on their functionality.
Figure~\ref{v8code}(a) and (b) show real examples of common patterns used
extensively throughout our code both to wrap CUDA structs and unwrap CUDA
structs for CUDA Driver API calls.  

Figure~\ref{v8code} (a) shows the code for creating a new connection to a
CUDA-enabled device. The user provides as input the device number and expects
the function to return a JavaScript Object Wrapper for a (CUDA object name).
MORE HERE...

Figure~\ref{v8code} (b) shows the code for freeing device memory.

As can be seen through the code examples above, our implementation is intuitive,
straightforward, and can be easily extended in the future as more features are
included within the \name specification.

%this should be in related work
%Another project we leaned heavily on for our implementation is
%node-cuda~\cite{nodecuda}. Aptly name, this project goal is to provde

\begin{itemize}

	\item general things about the CUDA Driver API? (maybe need more with this)

\end{itemize}




