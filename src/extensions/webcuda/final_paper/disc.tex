
\subsection{Challenges Encountered}
While our idea of adding CUDA bindings to JavaScript and the final
implementation is straightforward, there was much experimentation to get to this point.
One of the main obstacles encountered during this project is understanding the
V8 code base. The non-machine backend specific section of code is around 200K
LOC. In addition, there is not a definitive source explaining the compiler: some
blogs exist, but they oftentimes cover good programming practices for code to
performs well on the v8 compiler as opposed to explaining how the compiler works
itself. Therefore, it is necessary to look at the code itself to determine the
best way to interact with the compiler.

Luckily, two of the authors of the paper have had much experience working with
the v8 codebased and thus did not have such a steep learning curve. However, v8
is a ongoing project and is rapidly changing. Because of this, we wanted to work
on the most recent version to give our project the most relevance. The author's
previous experience was on a version of v8 from 2012. While the basic structure
of the compiler in both versions was similar, the authors were astonished to
find how many of the class and method names had changed. There was a major
overhaul of the external programmers API last summer, which has made most
information provided is blogs obsolete. Even Google's Embedder's Guide
Documentation \cite{embeddersGuide} does not correspond to the updated API.
Therefore, it took longer than expected to implement \name and caused many
headaches. Thankfully, with our newfound experience with this version of v8, any
additional changes necessary to be made to \name should be straightforward to
implement.

\subsection{WebCUDA Limitations}
While the current \name specification and implementation has enough
functionality for many CUDA implementations, many CUDA Runtime features do not
have a mapping in \namens. Below we highlight some limitations of \namens.

\subparagraph{Streaming Memory}
 One of the major limitations of \namens, as seen in
Section~\ref{eval}, is the lack of support for CUDA Streaming Memory
\cite{cuStream}. The authors felt Streaming Memory is not a core
CUDA feature and felt is was outside of the scope of this project.
Luckily, there is no fundamental reason CUDA Streams cannot be integrated into 
\name and can easily be added in future versions.

\subparagraph{Multiple Dimensioned Arrays}
As JavaScript implements multiple dimensioned arrays, as an array of potentially
non-contiguous arrays, it is not clear how to extend \name to include
support for multiple dimension array memory transfers between the host and
CUDA-enabled device. While this affects the ease of programmability, it does not
limit the range of programs which can be ported to \namens.

\subparagraph{Programmability} Many libraries, such as thrust~\cite{thrust} and
many math libraries~\cite{magma, cuSparse, arrayFire}, exist to simplify CUDA
programming. Unfortunately, these libraries are unavailable in the \name environment. While
this is a current setback, as CUDA programming in the browser becomes prevalent,
inevitably tools will be created to simplify the process GPGPU programming in
this domain.

In addition, the CUDA runtime is able to perform many actions necessary to
connect to a CUDA device without the programmers knowledge that the programmer
must explicitly perform in \namens, such as creating a CUDA context and
connecting to a device. \name can be extended to perform many of the same
monitoring features as the CUDA Runtime, but this would require a nontrivial
amount of effort to implement. The authors decided to leave this as future work,
as this does not limit the capabilities of \namens.

\subparagraph{Performance}
asynchronous calls


\subparagraph{The Great Unknown}

\subsection{Programmability}
want something here about how easy it is to write webcuda programs. (change section title accordingly)

\subsection{Security}
\subsection{Future Work}
