

\subsection{Challenges Encountered} \label{challenges} While our idea of adding
CUDA bindings to JavaScript and its implementation in V8 is straightforward,
we found the implementation to be more difficult than expected.  One of the main obstacles
encountered during this project was understanding the V8 code base. The
machine-independent backend section of compiler is around 200K LOC. In addition,
there is not a definitive source explaining the compiler: some blogs exist, but
they oftentimes focus on programming practices for inducing higher-performing
JavaScript code as opposed to explaining how the compiler itself works.
Therefore, it was necessary to look deep into the source code to determine the
best way to interact with the compiler.

V8 is an ongoing project and is rapidly improving and including new features.
Because of this, we wanted to work on its most recent version to give our
project the most relevance.  Two of the paper's authors have much experience
working with the V8 codebase and thus did not expect any difficulty implementing
the \name standard.  The author's previous experience was on a version of V8
from 2012.  While the basic structure of the compiler in both versions is
similar, the authors were astonished to find how many of V8's class and method
names had changed. There was a major overhaul of the external programmers API
last summer, which has made most information provided in blogs obsolete. Even
Google's Embedder's Guide documentation \cite{embeddersGuide} does not
correspond to the updated API.  Therefore, it was much more difficult and took
longer than expected to implement \namens. Thankfully, with our newfound
experience working on this version of V8, any additional changes necessary to be
made to \name should be straightforward to implement.

\subsection{WebCUDA Limitations}
While the current \name specification and implementation has enough
functionality for many CUDA applications, many CUDA Runtime features do not
have a mapping in \namens. Below we highlight some limitations of \namens.

\subparagraph{Streaming Memory} One of the major limitations of \name is the
lack of support for CUDA Streaming Memory. The authors felt Streaming Memory is
not a core CUDA feature and determined it to be outside of the scope of this project.
Luckily, there is no fundamental reason CUDA Streams cannot be integrated into
\name and this functionality may be added in future versions.

%Don't think this paragraph adds anything
%\subparagraph{Multiple Dimensioned Arrays}
%As JavaScript implements multiple dimensioned arrays, as an array of potentially
%non-contiguous arrays, it is not clear how to extend \name to include
%support for multiple dimension array memory transfers between the host and
%CUDA-enabled device. While this affects the ease of programmability, it does not
%limit the range of programs which can be ported to \namens.

\subparagraph{Programmability} 
Many libraries, such as thrust~\cite{thrust}, exist to simplify CUDA
programming. Unfortunately, these libraries are unavailable in the \name environment. While
this is a current setback, as CUDA programming in the browser becomes prevalent,
inevitably tools will be created to simplify the process GPGPU programming in
this domain.

In addition, the CUDA Runtime is able to perform many actions necessary to
connect to a CUDA device in the background that the programmer
must explicitly perform in \namens, such as connecting to a device and creating
a CUDA context. \name can be extended to perform many of the same
monitoring features as the CUDA Runtime, but this would require a nontrivial
amount of effort to implement. The authors decided to leave this as future work,
as this does not limit the capabilities of \namens.

\subparagraph{Performance}
%this is not necessary
%Our implementation does not currently allow asynchronous memory transfers
%transfers between the host and device. This hurts performance since the host
%device cannot be performing work while the transfer is in progress. While
%straightforward to include in the \name spec, the authors felt adding this
%feature was unnecessary. 
%
%As shown in Section~\ref{memSlow}, memory allocation for JavaScript's Typed Arrays
%is currently a big bottleneck in \namens. As discussed in Section~\ref{future},
%we plan to explore and optimize V8's implementation of instantiating Typed Arrays.
In all likelihood, there are additional CUDA features used to optimize
performance in the CUDA Runtime not present in \namens. Note that
while these hurt \namens's performance compared to CUDA, the main goal of \name
is to improve performance in the browser domain, which, as shown in
Section~\ref{eval}, \name successfully accomplishes.

\subparagraph{Cross-Domain Support} Since CUDA code can only run on NVidia
devices, \name cannot run on every platform. Therefore, the programmer must
create two versions of their code to account for whether their application is
running on a \namens-supported platform or not. This is common for other
features such as WebGL and to account for the various ways different browsers
support the DOM in JavaScript. The ideal solution is either NVidia extends their
compiler support to include non-NVidia products or \name is augmented with
support for running natively JavaScript programs on the GPU, as discussed in
Section~\ref{future}.

\subparagraph{The Great Unknown} Due to the author's lack of expertise in CUDA
programming, it is likely \name does not have some functionality experienced
CUDA programmers may deem essential. In these cases, the authors hope
programmers can extend \name in ways they see fit.  This paper provides
programmers the necessary insights into our implementation to allow them to
augment and improve our implementation. In addition, by releasing our extensions
to the public domain along with doxgyen~\cite{doxygen} and JSDoc~\cite{JSDOC}
documentation, we have tried to make extending \name as simple as possible.


%\begin{verbatim} think this section should not exist \end{verbatim}
%\subsection{Programmability} want something here about how easy it is to write
%webcuda programs. (change section title accordingly).

\subsection{\name vs WebCL}
\label{webCLDisc}
The Khronos Group, the consortium behind the OpenCL
\cite{openCL}, OpenGL \cite{openGL}, and many other specifications, has also
proposed a specification to bring OpenCL-like bindings to JavaScript named WebCL
\cite{webCLSpec}. This project is very similar to \namens, except the bindings are for
OpenCL instead of CUDA.

Like \namens, the actual GPU kernels must be written in OpenCL (vs CUDA in
\namens), and not in JavaScript. While it seems logical for the GPU kernels to
be written in JavaScript, as we talk about in Section~\ref{future}, OpenCL has
made decisions similar to the \name specification. One can rationalize this
decision from both a programmer's and developer's perspective. From the
developer's perspective, is it obviously easier to create wrappers to existing
calls in another language instead of having to parse and compile the JavaScript
kernels into CUDA or OpenCL executables. From the programmer's perspective, there
are also benefits for the GPU kernels remaining in OpenCL. The main advantage is
being able to port existing kernels written in the OpenCL domain directly to
WebCL without having to convert them to another language. In our experience, the
JavaScript binding code is very minimal so rewriting the host code is not a
large overhead, whereas GPU kernels can be very large and complicated.
Therefore, the porting burden is substantially lower with specification
decisions made by both WebCL and \namens.

While WebCL is very similar to our project, we believe it does not detract from
the merit of \namens. One reason is CUDA has more capabilities than OpenCL. Due
to \namens's connection with a specific platform, more hardware features are
able to be exposed to the programmer for higher performance. Another reason is
\name allows the vast collection of CUDA applications to be quickly ported to
the web domain. If a programmer were to use webCL, then the kernel would have to
be converted to openCL while in \name only the host operations need to be
rewritten in JavaScript.

Finally, \name is important because it is an another way of allowing web apps
to use the GPU to achieve performance gains. We feel that our project is
successful in encouraging conversation about the limitations of current JavaScript
execution speeds. It also demonstrates there are straightforward approaches to
improving the performance for some classes of applications by exposing GPUs to
web developers.


%While these project have similar goals, the implementation of each is very
%different due to the different backends.

\subsection{Security} \label{security} While one wishes web apps could leverage
the GPU freely throughout their execution, necessary security measures must be
taken to prevent unintended side effects. Currently CUDA applications can crash
one's computer.  The authors repeatedly inadvertently validated this fact
throughout the development and testing of \namens. This is a nightmare from a
security standpoint. While it is acceptable for a poorly written application to
crash, the typical CPU environment provides a level of isolation so that a
single user-level application, especially within the JavaScript, cannot easily
crash a computer. If and when browsers start to expose GPU extensions to web
apps, it will be unacceptable to allow any novice or malicious programmer to
crash a user's computer simply by the user navigating to their specific site.
The CUDA programming environment must be expanded to provide basic security
measures, such as runtime checks and forced kernel termination, before \name
can gain popularity.

We briefly explored the WebCL specification to see if WebCL addresses the
security issues addressed above. Their specification in fact makes security a
top priority. They split security measures into two categories: improper memory
accesses and Denial of Service.

Improper memory access protection handles situations when a program tries to
access areas of memory not belonging to the program. WebCL specifies that these
accesses during runtime must return the value ``0'' for reads and writes must be
discarded. They also have the WebCL Validator Project \cite{webclValidator} to
detect out of bounds-memory accesses at compile time. A WebCL program also must
not be able to read data left behind in allocated but not initialized memory.
Therefore, upon allocation, the WebCL specfication requires all allocated memory
to be initialized to zero.

The WebCL spec also addresses Denial of Service bugs, classified as when a long
running kernel causes the system to become unresponsive. In these cases, the
WebCL specification recommends WebCL implementations use the OpenCL 1.2
extensions and driver-level timers to detect and terminate contexts using an
unreasonable amount of the resources.  With these security measures, WebCL
should be safe for inclusion in web browsers. However, no current WebCL
prototype adheres to all security measures mentioned above \cite{nokiasecurity}.

While GPGPU computing is not yet commonplace in the browser, many browsers include
WebGL, which allows programmers to write GPU-enabled graphic applications in
JavaScript. Since they can use the GPU, many of the many same security concerns
arise in this domain. 

A quick survey found that the most recent WebGL specification also protects one
from the issues cited above.  However, in WebGL specification 1.0, security
flaws were found in the Cross-Origin Media security policy \cite{webGLerror1,
webGLerror2}. The Cross-Origin policy dictates how shaders and images/videos
from different sources can interact. The policy stipulated if an arbitrary
shader is used to read an image/video from another source, the resulting value
cannot be read.  However, researchers found the shader can be made in such a way
that the length of its runtime is dependent on the value of the pixel and hence
the image can be inferred. Recent specifications prevent these types of attacks
by requiring textures to be validated by Cross-Origin Resource Sharing (CROS).

Due to previous security issues, not all browsers have adapted and enabled WebGL
extensions. In Safari, the extension is disabled by default and Internet
Explorer has yet to fully implement the standard.

\subsection{Chromium Browser Integration} 
\label{chromeIntegration}
The end goal for \name is for it to be integrated into the
Chromium browser. For this project, however, we used d8, V8's standalone
JavaScript execution engine, in lieu of the actual browser. We chose to do this
to isolate the evaluation of \namens's JavaScript execution from the many other
tasks occurring in a web browser.  Another issue is the sheer size of the Chromium code.
The Chromium browser takes nearly 100GB and multiple hours to compile. Google
even has 
posts dedicated to recommending ways minimize the build overhead
\cite{linuxfasterbuilds}.  Since we had to compile often, the d8 environment,
where compilation times were often well under a minute, was much more practical.

Now that \name is relatively stable, porting the extension to Chromium should be
simple. The porting should be able to done in two steps. First, our extension
needs to be built alongside V8 in Chromium and a \texttt{webcuda} object must be
created during Chromium's initialization.  Next, sandboxing either needs to be
disabled or extended to allow for the necessary \name calls. Chromium sandboxes
all JavaScript execution for security reasons. This is to limit the scope of
interactions JavaScript applications can have with the system so malicious
programmers cannot access to sensitive material. We must adjust this security
policy to allow CUDA Driver API calls to pass through the sandboxed environment
and reach the proper drivers. 


\subsection{Future Work}
\label{future}
The obvious next steps for \name are to establish a security policy, as
discussed in Section~\ref{security} and to integrate our extension into the
Chromium browser. For security reasons, \name must enforce similar measures as
imposed by WebCL, namely having the runtime monitor memory accesses and
including the ability to terminate contexts using disproportionate amounts of
resources. Both security measures do not have clear solutions, since they involve
altering NVidia's tools and drivers. We hope that future NVidia products will
provide tools to create a more secure GPU programming environment.

Integrating \name into Chromium, as discussed in Section~\ref{chromeIntegration} should be
straightforward. The time constraints of this project are the only reason this
has yet to be done. We hope that in the future \name will be integrated into
Chromium to enable fair comparisons against WebCL.

Since the creation of Typed Arrays is currently a bottleneck in \namens, more
time must be spent exploring the related V8 code to determine sources
of overheads. Once this code is better understood, we hope to optimize Type
Array creation to reduce the performance gap between \name and CUDA.

More distant work includes enabling native JavaScript applications to be run on
the GPU. This involves altering the compiler to first recognize sections of JavaScript
code amenable to running on the GPU, next translate these code regions into GPU
executables, and finally execute them on the GPU. This will allow code
designed for \name to run on platforms without CUDA-enabled devices, since the
code is native JavaScript.  A recent work, ParallelJS~\cite{parallelJS}, has
created a runtime framework to execute JavaScript on the GPU and is discussed in
Section~\ref{related}.


