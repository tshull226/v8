
JavaScript is one of the top 10 most popular programming languages, 
considered by many as the “assembly language of the internet.” JavaScript is
prevalent in the vast majority of websites due to it being the only
ubiquitous, cross-platform scripting language available on the browser. The 
burgeoning mobile device market has increased the importance of JavaScript due
to the increased popularity of web applications
%The growing
%mobile device market will continue to boost the popularity of web applications.
Due to the importance of JavaScript, major web browser providers competitively
enhance their JavaScript engines. These improvements in JavaScript performance
have allowed for a proliferation in both the number of web applications and
their capabilities.  

However, currently JavaScript's single-threaded execution model fundamentally limits
performance and hinders adoption for computationally intensive and rich visual
computing applications on the web, such as 3D games, video processing
applications, and augmented reality applications. Such applications have much data
parallelism and can benefit from processing elements available in today’s
heterogeneous systems, such as Graphic Processing Units (GPUs).  Furthermore,
JavaScript is gaining attention at the server-side as well through popular
projects such as NodeJS \cite{nodeJS}, where operations are often
computationally intensive or data-parallel, expanding the JavaScript application
domain which can benefit from heterogeneous systems.

We propose to implement a JavaScript language extension to allow for bindings to
CUDA to provide portable and efficient access to CUDA-enabled GPUs from web
applications.  We call this project \name, highlighting the enabling of CUDA
programming from within the browser domain. Our contributions in this paper are
as follows:

%In addition to evaluating the performance impact using the real hardware, we
%plan to integrate the JavaScript engine with a heterogeneous system simulator to
%experiment with various architectural alternatives starting with the memory
%model of GPUs. We envision this to be the first steps towards a long-term goal
%of creating heterogeneous systems specialized for JavaScript execution.  

\begin{itemize}

\item Create JavaScript extensions to allow CUDA bindings within JavaScript
\item Compare performance of native CUDA, CUDA-enabled JavaScript, and native JavaScript benchmarks
\item Perform analysis of the effectiveness of the V8 runtime handler
\item Provide insights on the flexibility of adding extensions into V8/Chrome

\end{itemize}

Our evaluations show that \name is about to achieve a speedup of XX\% over
JavaScript Applications and are within XX\% of native CUDA performance.
Providing web developers access to high-performance graphics and
parallel computation will not only enable more applications to be ported to the
web environment, but it will also unleash a wave of creativity that will result
in new innovative web applications.

This paper is organized as follows: Section~\ref{background} provides background about JavaScript and CUDA, 
Sections \ref{overview} and \ref{imp} describe the structure and implementation of WebCUDA, and
Section~\ref{eval} demonstrates the performance benefits of WebCUDA. Finally, Section~\ref{disc}
provides discussion about our project, Section~\ref{related} highlights related
work, and Section~\ref{conc} concludes.

