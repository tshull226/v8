%(
JavaScript is one of the most popular programming languages~\cite{top10}, 
considered by many as the ``assembly language of the Internet.'' JavaScript is
prevalent in the vast majority of websites due to it being the only
ubiquitous, cross-platform scripting language available on the browser. The 
burgeoning mobile device market has also increased the importance of JavaScript due
to its contribution to the growing  popularity of web applications.
%The growing
%mobile device market will continue to boost the popularity of web applications.
Due to the importance of JavaScript, major web browser providers competitively
enhance their JavaScript engines. These improvements in JavaScript performance
have allowed for a proliferation in both the number of web applications and
their capabilities.  

However, currently JavaScript's single-threaded execution model fundamentally limits
performance and hinders adoption for computationally intensive and rich visual
computing applications on the web, such as 3D games, video processing
applications, and augmented reality applications. Such applications have much data
parallelism and can benefit from processing elements available in today’s
heterogeneous systems, such as Graphic Processing Units (GPUs).  Furthermore,
JavaScript is gaining attention on the server-side as well through popular
projects such as NodeJS \cite{nodeJS}, where operations are often
computationally intensive or data-parallel, expanding the JavaScript application
domain which can benefit from heterogeneous systems.

We propose a JavaScript language extension for CUDA bindings, providing portable
and efficient access to CUDA-enabled GPUs from web applications.  We call this
project \namens, highlighting the enabling of CUDA programming from within the
browser domain. Our contributions in this paper are as follows:

%In addition to evaluating the performance impact using the real hardware, we
%plan to integrate the JavaScript engine with a heterogeneous system simulator to
%experiment with various architectural alternatives starting with the memory
%model of GPUs. We envision this to be the first steps towards a long-term goal
%of creating heterogeneous systems specialized for JavaScript execution.  

\begin{itemize}

\item Define JavaScript extension to allow CUDA bindings within JavaScript code
\item Implement \name specification within a state of the art JavaScript compiler
\item Compare the performance of native CUDA, CUDA-enabled JavaScript, and native JavaScript benchmarks
\item Analyze overheads of the \name runtime.
\item Provide insights on the flexibility of adding extensions into V8

\end{itemize}

Our evaluations show that \name is able to achieve a 9x speedup on average over
JavaScript applications without negatively impacting native JavaScript performance.
Providing web developers access to high-performance, parallel computational
resources will not only enable more existing applications to be ported to the
web environment, but will also unleash a wave of creativity resulting in new
innovative web applications.

This paper is organized as follows: Section~\ref{background} provides background
about JavaScript and CUDA; Sections \ref{overview} and \ref{imp} describe the
structure and implementation of \namens; Section~\ref{eval} demonstrates the
performance benefits of \namens; Section~\ref{disc} provides discussion
about our project; Section~\ref{related} highlights related work; and
Section~\ref{conc} concludes.

